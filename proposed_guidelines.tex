\section{Proposed guidelines}\label{s:guidelines}

To assist in the establishment a minimum standard for the communication of computational results, it is proposed that the following could be inserted into the author and reviewer guidelines of journals in the weather and climate sciences. In places the language borrows from the guidelines recently adopted by \textit{Nature} \citep{Nature2014} and it is anticipated that the journal would provide links to examples of well documented computational results to assist both authors and reviewers in complying with these guildelines.

\subsection{Author guidelines}

If computer code is central to any of the paper's major conclusions, then the following is required as a minimum standard: 
\begin{enumerate}
\item A statement describing whether that code is available and setting out any restrictions on accessibility. Best practice involves managing code with a version control system like Git, Subversion or Mercurial, which is then linked to a publically accessible online repository like GitHub or Bitbucket. Authors are not expected to produce a brand new repository to accompany their paper - an `everyday' repository which also contains code that is not relevant to the paper is fine.  
\item A section within the paper that provides (i) a high-level description of the software used to execute that code (including citations for any academic papers written to describe that software), and (ii) details regarding the precise version of the software used listed in the following format: name, version number, release date, institution and DOI or URL.
\item A supplementary log file for each major result (including key figures) showing all computational steps taken from the initial download/attainment of the data to the final result. Best practice includes the time/date each step was executed as well as the unique revison number (or hash value) indicating which version of the code repository was used.
\end{enumerate}

Any practical issues preventing code sharing will be evaluated by the editors, who reserve the right to decline a paper if important code is unavailable. Authors should note that they are not obliged to support the reviewers or readers of this journal in repeating their computations.

\subsection{Reviewer guidelines}

The reviewer guidelines for most journals already ask if the methodology is explained in sufficient detail so that the paper's scientific conclusions could be tested by others. Such a statement could simply be added to as follows: "If computer code is central to any of those conclusions, then reviewers should ensure that the authors have attained the minimum standards outlined in the author guidelines. It should be noted that reviewers are not obliged to assess or execute the code associated with a submission. They must simply check that it is adequately documented."   