Weather and climate science has undergone a computational revolution in recent decades, to the point where all modern research relies heavily on software and code. Despite this profound change in the research methods employed by weather and climate scientists, the reporting of computational results has changed very little in relevant academic journals. This lag has led to something of a reproducibility crisis, whereby it is impossible to replicate and verify most of today's published computational results. While it is tempting to simply decry the slow response of journals and funding agencies in the face of this crisis, there are very few (if any) examples of reproducible weather and climate research upon which to base new communication standards. In an attempt to address this deficiency, this essay describes a procedure for reporting computational results that was employed in a recent \textit{Journal of Climate} paper. The procedure was developed to be consistent with recommended computational best practices and seeks to minimize the time burden on authors, which has been identified as the most important barrier to publishing code. It should provide a starting point for weather and climate scientists looking to publish reproducible research, and it is proposed that journals could adopt the procedure as a minimum standard.

  
  
  