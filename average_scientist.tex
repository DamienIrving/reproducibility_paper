\textbf{Box 1: The regular scientist}

For the sake of argument, I have attempted to describe a regular research scientist working in the weather/climate sciences. The characterisation is based on the few documented surveys of how computational scientists do their work \citep{Hannay2009,Stodden2010}, editorials describing current computational practices \citep[e.g.][]{Easterbrook2014} and my own personal experiences teaching computational best practices at numerous Software Carpentry workshops over the past few years while also working/studying in the weather and climate sciences. This regular scientist:
\begin{itemize}
\item Works with publicly available data (e.g. CMIP data, reanalysis data, observational data from a national weather service) that is often not trivial in size (e.g. it might be tens or hundreds of gigabytes), but is not so large as to be considered `big data' 
\item Aquired the knowledge to develop and use scientific software primarily from peers and through self-study, as opposed to formal education and training
\item Relies primarily on software like Python, MATLAB, IDL, NCL or R which has a large user/support base and is relatively simple to install on a Windows, Mac or Linux computer
\item Does most of their work on a desktop or intermediate computer (as opposed to a supercomputer)
\item Is only writing software/code for a specific task/article and is not looking for community uptake  
\item Works on their own or in a very small team (i.e. 2-3 other scientists) which does not have access to professional software developers for support
\end{itemize}

Some scientists might be regular in most but not all aspects of their work (e.g. all the points above might apply except they occasionally use a highly specialised software package that does not have a large support base) so this characterisation can be thought of as a baseline or minimum level of computation that essentially all weather and climate scientists are engaged in.  

%Their code would produce the same result whether they ran it on their Mac OS X laptop at home or a Microsoft Windows machine in their office, meaning details like the precise computing environment are also not critical.

% Complexity of task: climatologies, anomalies, fluxes, correlations





