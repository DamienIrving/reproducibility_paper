\section{Introduction}
% First two sentences copied from Ince2012
The rise of computational science has led to unprecedented opportunities in the weather and climate sciences. Ever more powerful computers enable theories to be investigated that were thought almost intractable a decade ago, while new hardware technologies allow data collection in the most inhospitable environments. In order to analyse the vast quantities of data now available to them, modern practitioners – most of whom are not computational experts – use an increasingly rich and diverse set of software tools and packages. Today's weather or climate scientist is far more likely to be found debugging code written in Python, MATLAB, IDL, NCL or R, than to be pouring over satellite images or releasing radiosondes. 
%--
This computational revolution is not unique to the weather and climate sciences, and has led to something of a reproducibility crisis in published research. As numeroous commentators have pointed out over the past few years, it is impossible to replicate and thus verify most of the computational results presented in journal articles today.

