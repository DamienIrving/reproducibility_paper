\textbf{Box 3: Example log file}

The following is the log file corresponding to Figure \ref{fig:example_figure}. It shows every command line entry used in creating the figure from the initial download of the underlying data to the final generation of the figure.  

\begin{lstlisting}[language=bash] 
Sat Nov 29 06:09:28 2014: /usr/local/anaconda/bin/python /home/STUDENT/dbirving/phd/visualisation/plot_hilbert.py /mnt/meteo0/data/simmonds/dbirving/ERAInterim/data/va_ERAInterim_500hPa_030day-runmean_native.nc va hilbert_zw3_w19_va_ERAInterim_500hPa_030day-runmean_native-55S_1986-05-22_2006-07-29.png 1 2 --latitude -55 --dates 1986-05-22 2006-07-29 --wavenumbers 1 9 --figure_size 15 6 (Git hash: 8a42fff) 

Fri Nov 28 16:36:21 2014: ncatted -O -a axis,time,c,c,T /mnt/meteo0/data/simmonds/dbirving/ERAInterim/data/va_ERAInterim_500hPa_030day-runmean_native.nc

Fri Nov 28 16:25:18 2014: cdo runmean,30 /mnt/meteo0/data/simmonds/dbirving/ERAInterim/data/va_ERAInterim_500hPa_daily_native.nc /mnt/meteo0/data/simmonds/dbirving/ERAInterim/data/va_ERAInterim_500hPa_030day-runmean_native.nc

Mon Nov 10 17:15:49 2014: ncatted -O -a level,va,o,c,500hPa va_ERAInterim_500hPa_daily_native.nc

Mon Nov 10 16:31:05 2014: ncatted -O -a long_name,va,o,c,northward_wind va_ERAInterim_500hPa_daily_native.nc

Thu Aug 21 10:57:46 2014: ncatted -O -a axis,time,c,c,T va_ERAInterim_500hPa_daily_native.nc

Thu Aug 21 10:34:46 2014: ncrename -O -v v,va va_ERAInterim_500hPa_daily_native.nc

Thu Aug 21 10:26:49 2014: cdo invertlat -sellonlatbox,0,359.9,-90,90 -daymean va_ERAInterim_500hPa_6hourly_native.nc va_ERAInterim_500hPa_daily_native.nc

Thu Aug 21 10:14:59 2014: cdo mergetime ../download/va_ERAInterim_500hPa_6hourly-1979-1988_native_unpacked.nc ../download/va_ERAInterim_500hPa_6hourly-1989-1998_native_unpacked.nc ../download/va_ERAInterim_500hPa_6hourly-1999-2008_native_unpacked.nc ../download/va_ERAInterim_500hPa_6hourly-2009-2014_native_unpacked.nc va_ERAInterim_500hPa_6hourly_native.nc 

Thu Aug 21 10:13:35 2014: ncpdq -P upk va_ERAInterim_500hPa_6hourly-2009-2014_native.nc va_ERAInterim_500hPa_6hourly-2009-2014_native_unpacked.nc

2014-08-20 23:16:22 GMT: Download of ERA_Interim, 6 hourly, 500hPa meridional wind (va) data 
\end{lstlisting}

The most important feature of this log file is that besides a slight amendment to the initial download entry (the default text provided by the ERA-Interim data server was not particularly self explanatory), no manual editing of its contents has been done. This means that if a reviewer asks for a slight modification to the figure, the regeneration of a new log file is trivial. By resisting the urge to clean up the file (e.g. one might consider removing path details like \verb|/mnt/meteo0/data/simmonds/dbirving/ERAInterim/data/|) it doubles as a record that is highly useful to the author in retracing their own steps (e.g. they can use it to recall where they stored the output data on their local machine). Other features of note include:
\begin{itemize}
\item Since it cannot be assumed that the latest version of any code repository was used to generate all the results in a paper, the unique revision number / hash value is given whenever a script written by the author was used.
\item When more than one input file is passed to an NCO or CDO function, the history of only one of those files is retained in the output file. On occasions where this is not appropriate (i.e. where the history of the multiple input files are very different), it is important to ensure that the history of all input files is retained going forward. There are a number of examples of this in the log files provided by \citet{Irving2015}. 
\end{itemize}
