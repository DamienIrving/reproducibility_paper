\section{A possible solution}

An example of what a baseline standard for the communication of computational results might look like in the weather and climate sciences was recently published in the \textit{Journal of Climate} \citep{Irving2015}. It comprises three main components: a short computation section within the paper, an externally hosted snapshot of the associated code repository and the provision of supplementary log files that capture the data processing steps taken in producing each key result. 

\subsection{Computation section}

The first requirement of the proposed communication standard is the inclusion of a short computation section (most likely within or immediately before or after the traditional methods section). The suggested format for that section is based on the recommendations of \citet{Jackson2012}, who point out that there is a    

 The key thing is that there is a difference between describing the software that was used and citing it. 
 
It opens with a general, high-level description of the software used, including citations to any papers that have been written about the software (e.g. in JORS) or just what they tell you to cite on their website. This is a nice thing to do to give academic credit where it's due (and also to give the reader a link to a nice high level summary of what the software does), but these citations aren't sufficient to document exactly which version you used, so I then went with a bullet point list with all the details, using the format suggested in Section 6.4 of \citet{Jackson2012}. 

\subsection{Snapshot of code repository}

Needs to be consistent with coding best practices \citep{Wilson2014a}

\subsection{Log files}

Blah.