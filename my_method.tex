\section{A possible solution}

An example of what a baseline standard for the communication of computational results might look like in the weather and climate sciences was recently published in the \textit{Journal of Climate} \citep{Irving2015}. It comprises three main components: a short computation section within the paper, an externally hosted snapshot of the associated code repository and the provision of supplementary log files that capture the data processing steps taken in producing each key result. 

\subsection{Computation section}

The first requirement of the proposed communication standard is the inclusion of a short computation section (most likely within or immediately before/after the traditional methods section). The suggested format for that section is based on the recommendations of \citet{Jackson2012}, who point out that there is an important difference between describing the software that was used (i.e. in enough detail to convey exactly which version) and citing it (i.e. so that the authors get appropriate academic credit). As such, the computation section should consist of two distinct parts. The first is a high-level description of the software used, including citations to any papers that have been written about the software. Authors of scientific software are increasingly publishing with journals like the \textit{Journal of Open Research Software}, so it is important to give them a citation (i.e. academic credit) where it's due. This description is also useful in briefly articulating what each software item is actually used for.

The second part of the computation section should provide more specific details regarding the precise version of the software used and the environment in which it was run. These details should be listed in the following format: name, version number, release date, institution and DOI or URL. Finally, the computational section should also provide a link to a snapshot of the code (Section X) referred to in the supplementary log files (Section X). An example computation section is shown in Box 2.  
