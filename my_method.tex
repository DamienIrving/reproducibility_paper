\section{A possible solution}

An example of what a baseline standard for the communication of computational results might look like in the weather and climate sciences was recently published in the \textit{Journal of Climate} \citep{Irving2015}. It comprises three main components: a short computation section within the paper, an externally hosted snapshot of the associated code repository and the provision of supplementary log files that capture the data processing steps taken in producing each key result. 

\subsection{Computation section}

The first requirement of the proposed communication standard is the inclusion of a short computation section (most likely within or immediately before/after the traditional methods section). The suggested format for that section is based on the recommendations of \citet{Jackson2012}, who point out that there is an important difference between describing the software that was used (i.e. in enough detail to convey exactly which version) and citing it (i.e. so that the authors get appropriate academic credit). As such, the computation section should consist of two distinct parts. The first is a high-level description of the software used, including citations to any papers that have been written about the software. Authors of scientific software are increasingly publishing with journals like the \textit{Journal of Open Research Software}, so it is important to give them a citation (i.e. academic credit) where it's due. This description is also useful in briefly articulating what each software item is actually used for.

The second part of the computation section should provide more specific details regarding the precise version of the software used and the environment in which it was run. These details should be listed in the following format: name, version number, release date, institution and DOI or URL. Finally, the computational section should also provide a link to a snapshot of the code (Section \ref{s:snapshot}) referred to in the supplementary log files (Section \ref{s:log_files}). An example computation section is shown in Box 2.  

\subsection{Snapshot of code repository}\label{s:snapshot}

It is important that any communication standard for code (a) does not substantially increase the workload of authors, and (b) is consistent with established best practices for scientific computing \citep{Wilson2014a}. With respect to the latter, it would not be appropriate to ask authors to provide a single script for each key figure or analysis, since an important practice for reducing bugs/errors is to modularize code rather than copying and pasting. This means that authors should be expected/encouraged to provide a whole library of code (i.e. a repository containing many interconnected scripts).

A related computational best practice is the use of a version control system like Git or Subversion. These systems can easily be linked to an online hosting service such as GitHub or Bitbucket, which is the means by which a code repository can be made publically available. Authors will often produce a pristine GitHub or Bitbucket repository for a paper they have submitted (i.e. one that contains only the code that is directly relevant to that paper), however this should not be an expectation. Not only is this a time consuming practice that would likely involve a degree of cutting and pasting, it also goes against the workflow that version control promotes. By tagging an offical release of their everyday repository, authors can signal to readers what version of their code was used to produce the results in their paper,    


Code as a research object.
Privacy concerns.


\subsection{Log files}\label{s:log_files}

Blah.