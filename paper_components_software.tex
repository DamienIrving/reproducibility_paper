\subsection{Components}

At first glance, the only difference between a regular journal article and that of \citet{Irving2016} is the addition of short computation section (Box 2). This section accompanies the usual description of data and methods within an article and briefly cites the major software packages used in the research, before pointing the reader to three key supplementary items: (1) a more detailed description of the software used, (2) a version controlled and publicly available code repository and (3) a collection of supplementary log files that capture the data processing steps taken in producing each key result. The code repository is hosted at GitHub (a popular code sharing website), while the detailed software description and log files are hosted at Figshare \citep{Irving2015}, which is a site where researchers commonly archive the `long tail' of their research (e.g. supplementary figures, code and data). 

\subsubsection{Software description}

There is an important difference between citing the software that was used in a study (i.e. so that the authors get appropriate academic credit) and describing it in sufficient detail so as to convey the precise version and computing environment \citep{Jackson2012}. Recognising this, \citet{Irving2016} began their computation section with a high-level description of the software used, which included citations to any papers written about the software. Authors of scientific software are increasingly publishing with journals like the \textit{Journal of Open Research Software}, so it is important for users of that software to cite those papers within their manuscripts. This description is also useful for briefly articulating what general tasks each software item was used for (e.g. plotting, data analysis, file manipulation). Such an overview does not provide sufficient detail to recreate the computing environment used in the study, so \citet{Irving2016} provide a link to a supplementary file on Figshare that documents the precise version of each software package used and the operating system upon which it was run (namely the name, version number, release date, institution and DOI or URL). 


 
  
  