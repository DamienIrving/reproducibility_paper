\section{Conclusion}

In order to combat the reproducibility crisis in published computational research, a simple procedure for communicating computational results has been demonstrated \citep{Irving2015} and its rationale discussed. The procedure involves three key additions to traditional scientific papers: (1) a short computation section within the paper, (2) the availability of a (preferably version controlled and publicly accessible) code repository and (3) the provision of supplementary log files that capture the data processing steps taken in producing each key result. It should provide a starting point for weather and climate scientists (and perhaps computational scientists more generally) looking to publish reproducible research, and could be adopted as a minimum standard by relevant academic journals.

The procedure/standard was developed to be consistent with recommended computational best practices and seeks to minimize the time burden on authors, as this has been identified as the most important barrier to publishing code. In particular, best practice dictates that at a minimum weather and climate scientists should be (a) writing data analysis scripts so they can re-run their analyses, (b) using version control to manage those scripts for backup and ease of sharing/collaboration and (c) storing the details of their analysis steps in the global history attribute of their netCDF data files (or following an equivalent process for other file formats) to ensure the complete provenance of their data. In order to make their published results reproducible, it follows that the minimum an author would need to do is simply make those history attributes available (in what I am calling log files) along with the associated code repository and a description of the software used to execute that code. In my experience as a Software Carpentry instructor, most weather and climate scientists are comfortable with the idea of scripting, but very few use version control and almost none keep track of the provenance of their data. The attainment of this minimum standard would therefore involve changes to the workflow of regular weather and climate scientists, however the standard has been designed to only require skills that scientists already have or can attain relatively easily.  

As in many other aspects of life, if everyone just followed the minimum standard things wouldn't be that great. By way of analogy, minimum standards in the construction industry ensure that buildings won't fall over or otherwise kill their inhabitants, but if everyone only built to the minimum standard towns and cities would be hugely energy inefficient. The proposed minimum standard ensures that published results work are reproducible (which is a massive improvement on the current state of affairs), but it would be a lot of work for readers to come along and recreate a complex workflow from those log files. Authors should seek to go beyond the minimum standards in order to improve the comprehensibility of their published computational results (e.g. use RunMyCode and/or package their software so it can be installed via pip or binstar/conda) just like builders should go for a 5-star energy efficiency rating.
