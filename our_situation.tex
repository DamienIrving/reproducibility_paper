\section{How did it come to this?}

In assigning blame for the reproducibility crisis, it would be easy to focus on the weak (or absent) requirements that funding agencies and publishers place upon authors. While strengthened requirements could certainly help, the required cultural change may not be best achieved by simply forcing authors to change their ways (REF FROM ONE OF THOSE ROUNDTABLE MEETINGS?). In my experience most weather and climate scientists are sympathetic to ideals like open science and reproducibility (FIND A SURVEY WITH ACTUAL DATA), but the practicalities of those ideals seem too difficult, particularly when the pressure to publish is so strong and unrelenting. 

There are a few reasons why an individual scientist might place reproducibiliy in the `too hard' basket. The first is that many of the suggested best practices and tools out there are unecessarily elaborate for a regular scientist. Data proverance tracking systems like VisTrails \citep{Freire2012} and software environment managers like Docker and Vagrant \citep{Stodden2014} are fantastic tools for small teams of software engineers or very experienced scientific programmers dealing with very large workflows (e.g. post-processing of CMIP5 model runs) and/or production style code (e.g. a climate model or satellite retrieval algorithm that has high re-use potential in the wider community), but regular scientists have neither the requisite computational experience or research problems of sufficient scale and complexity to make use of such tools.      

As pointed out by \citet{Easterbrook2014}, regular scientits they are writing code for a specific task (usually analysing CMIP or raanalysis data) and are not looking for uptake 

The tips and ideas that people are throwing around are for production style code (e.g. models, satellite retrieval algorithms). Your garden variety researcher is not doing that stuff - as \citet{Easterbrook2014} points out, they are writing code for a specific task (usually analysing CMIP or raanalysis data) and are not looking for uptake 



The other is a lack of confidence/skills. While weather and climate scientists spend much of their time writing code, most have had little or no formal training \citet{Hannay2009}.   


We need a simple example to follow for the everyday scientist.