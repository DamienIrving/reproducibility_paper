\section{How did it come to this?}

In assigning blame for the reproducibility crisis, it would be easy to focus on the weak (or absent) requirements that funding agencies and publishers place upon authors. While strengthened requirements could certainly help, the required cultural change may not be best achieved by simply forcing authors to change their ways (REF FROM ONE OF THOSE ROUNDTABLE MEETINGS?). In my experience most weather and climate scientists are sympathetic to ideals like open science and reproducibility (FIND A SURVEY WITH ACTUAL DATA), but the practicalities of those ideals seem too difficult, particularly when the pressure to publish is so strong and unrelenting. 

There are a few reasons why an individual scientist might place reproducibiliy in the too hard basket. The first is that most of the guidelines out there are written for high level users 

The other is a lack of confidence/skills. While weather and climate scientists spend much of their time writing code, most have had little or no formal training \citet{Hannay2009}.   


We need a simple example to follow for the everyday scientist.