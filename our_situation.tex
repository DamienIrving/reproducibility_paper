\section{How did it come to this?}

In assigning blame for the reproducibility crisis, it would be easy to focus on the weak (or absent) requirements that funding agencies and publishers place upon authors. While they could certainly do more to help, the required cultural change may not be best achieved by simply forcing authors to change their ways (REF FROM ONE OF THOSE ROUNDTABLE MEETINGS?). In my experience weather and climate scientists are typically very sympathetic to ideals like open science and reproducibility, but the practicalities of those ideals seem to difficult, particularly when the pressure to publish is so strong and unrelenting. Part of that mindset probably stems from the fact that while weather and climate scientists spend a lot of their time writing code, most hve had little or no formal training. 


Most of the recommendations and guidance out there focus 
Computational competence is low (that's why SWC is so popular), which means people aren't super confident. 

We need a simple example to follow for the everyday scientist.