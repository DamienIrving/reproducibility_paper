\section{How did it come to this?}

In assigning blame for the reproducibility crisis, it would be easy to focus on the weak (or absent) requirements that funding agencies and publishers place upon authors. While strengthened requirements could certainly help in motivating scientists to change their ways, most scientists are already sympathetic to ideals like open science and reproducibility. The major factor holding them back is therefore not a lack of motivation, but rather that the practicalities of those ideals seem too difficult and time consuming, particularly when the pressure to publish is so strong and unrelenting \citep[e.g.][]{Stodden2010}. 

There are a couple of reasons why an individual working in the weather and climate sciences might place reproducibiliy in the `too hard' basket. The first is the bewildering array of suggested tools and best practices out there. An appropriate solution for any given scientist no doubt exists within that collection of, but it is obscured  for any given    many of the suggested best practices and tools out there are unecessarily elaborate for a regular scientist. Data proverance tracking systems like VisTrails \citep{Freire2012} and PyRDM \citep{Jacobs2014} and software environment managers like Docker and Vagrant \citep{Stodden2014} are fantastic tools for small teams of software engineers or very experienced scientific programmers dealing with very large workflows (e.g. post-processing of thousands of CMIP5 model runs), complex model simulations (e.g. coupled climate models) and/or production style code (e.g. a satellite retrieval algorithm that has high re-use potential in the wider community), but regular scientists have neither the requisite computational experience or research problems of sufficient scale and complexity to make use of such tools.      

As pointed out by \citet{Easterbrook2014}, regular scientits are typically only writing code for a specific task/article and are not looking for community uptake. That code would produce the same result whether they ran it on their Mac OS X laptop at home or a Microsoft Windows machine in their office, meaning the details of the precise computing environment are also not critical.

The other is a lack of confidence/skills. While weather and climate scientists spend much of their time writing code, most have had little or no formal training \citet{Hannay2009}.   


We need a simple example to follow for the everyday scientist.